\documentclass{ufdissertation}\sloppy

%%%%%%%%%%%%%%%%%%%%%%%%%%%%%%%%%%%%%%%%%%%%%%%%%%%%%%%%%%%%%%%%%%%%%%%%%%%%%%%%
%%%                 User Package and Style File loading.
%%%%%%%%%%%%%%%%%%%%%%%%%%%%%%%%%%%%%%%%%%%%%%%%%%%%%%%%%%%%%%%%%%%%%%%%%%%%%%%%

%\usepackage{CustomMacros}%  This is a user macro/style file.

\usepackage{tikz}%       tikz is used by almost everyone, but certainly by me for this.
\usepackage{pgfplots}%   pgfplots is tikz but better.
\pgfplotsset{compat=1.7}
\usepackage{booktabs}
\usepackage{makecell}
%\usepackage{amsrefs}%   amsrefs contains the .bibtex style content for mathematician papers.


%%%%%%%%%%%%%%%%%%%%%%%%%%%%%%%%%%%%%%%%%%%%%%%%%%%%%%%%%%%%%%%%%%%%%%%%%%%%%%%%
%%%                     User Configuration commands
%%%%%%%%%%%%%%%%%%%%%%%%%%%%%%%%%%%%%%%%%%%%%%%%%%%%%%%%%%%%%%%%%%%%%%%%%%%%%%%%

%% Uncomment the relevant line below if you have tables or figures.
\haveTablestrue%        Uncomment this if you have tables in your thesis.
\haveFigurestrue%       Uncomment this if you have figures in your thesis.
%\haveObjectstrue%       Uncomment this if you have Objects in your thesis. This is almost certainly not the case however.

%%%%%%%%%%%%%%%%%%%%%%%%%%%%%%%%%%%%%%%%%%%%%%%%%%%%%%%%%%%%%%%%%%%%%%%%%%%%%%%%
%%% Below are the commands to set the degree type, department, graduation time, and chair.
%       Most of these are self explanatory.
%       Note: The \chair command takes an optional argument for a cochair.
%           So if John was your chair and Jacob was a cochair, you would use \chair[Jacob]{John}.
%           If John was your chair and you had no cochair, you can simply use \chair{John}.
%%%%%%%%%%%%%%%%%%%%%%%%%%%%%%%%%%%%%%%%%%%%%%%%%%%%%%%%%%%%%%%%%%%%%%%%%%%%%%%%

\title{$title$}%  Put your title here.

\degreeType{Doctorate of Philosophy}%   Official name of your degree; eg "Doctorate of Philosophy".
\major{$department$}%                    Your official Department
\author{$author$}%                  Your Name
\thesisType{Dissertation}%              Dissertation (PhD) or Thesis (Masters)
\degreeYear{$date$}%                      Intended graduation year (not the year you submit the thesis)
\degreeMonth{$month$}%                   Month of graduation should be May, August, or December.
\chair[$altadvisor$]{$advisor$}%                   Chair and Cochair (see comment block above).


%%%%%%%%%%%%%%%%%%%%%%%%%%%%%%%%%%%%%%%%%%%%%%%%%%%%%%%%%%%%%%%%%%%%%%%%%%%%%%%%
%%% For each of the following, type in the name of the file that contains each section.
%       They are assumed to be tex files, but if they aren't the command takes an optional argument for the extension.
%       So, you could load dedication.tex as your dedication file using \setDedicationFile{dedication}
%       You could load dedication.txt instead with \setDedicationFile[txt]{dedication}.
%       NOTE: For some compilers they may or may not add a .tex to the end of the file automatically.
%           If you get a "couldn't find dedication.tex.tex" type error, try the command with an empty optional argument,
%           e.g. \setDedicationFile[]{dedication}
%%%
%%%%%%%%%%%%%%%%%%%%%%%%%%%%%%%%%%%%%%%%%%%%%%%%%%%%%%%%%%%%%%%%%%%%%%%%%%%%%%%%

%%% These are REQUIRED sections; easiest to do via these commands.

\setDedicationFile{dedicationFile}%                 Dedication Page
\setAcknowledgementsFile{acknowledgementsFile}%     Acknowledgements Page
\setAbstractFile{abstractFile}%                     Abstract Page (This should only include the abstract itself)
\setReferenceFile{referenceFile}{amsplain}%         References. First argument is your bibtex source file
%                                                       the second argument is your bibtex style file.
\setBiographicalFile{biographyFile}%                Biography file of the Author (you).

%%% These are NOT required, so only use them if you actually need/have them.

\setAbbreviationsFile{abbreviations}%           Abbreviations Page
\setAppendixFile{appendix}%                     Appendix Content; hyperlinking might be weird.
\multipleAppendixtrue%                          Uncomment this if you have more than one appendix,
%                                                   comment it if you have only one appendix.


%%%%%%%                     End of File Assignment
%%%%%%%%%%%%%%%%%%%%%%%%%%%%%%%%%%%%%%%%%%%%%%%%%%%%%%%%%%%%%%%%%%%%%%%%%%%%%%%%

% From {rticles}
$if(csl-refs)$
\newlength{\csllabelwidth}
\setlength{\csllabelwidth}{3em}
\newlength{\cslhangindent}
\setlength{\cslhangindent}{1.5em}
% for Pandoc 2.8 to 2.10.1
\newenvironment{cslreferences}%
{$if(csl-hanging-indent)$\setlength{\parindent}{0pt}%
	\everypar{\setlength{\hangindent}{\cslhangindent}}\ignorespaces$endif$}%
{\par}
% For Pandoc 2.11+
\newenvironment{CSLReferences}[3] % #1 hanging-ident, #2 entry spacing
{% don't indent paragraphs
	\setlength{\parindent}{0pt}
	% turn on hanging indent if param 1 is 1
	\ifodd #1 \everypar{\setlength{\hangindent}{\cslhangindent}}\ignorespaces\fi
	% set entry spacing
	\ifnum #2 > 0
	\setlength{\parskip}{#2\baselineskip}
	\fi
}%
{}
\usepackage{calc} % for calculating minipage widths
\newcommand{\CSLBlock}[1]{#1\hfill\break}
\newcommand{\CSLLeftMargin}[1]{\parbox[t]{\csllabelwidth}{#1}}
\newcommand{\CSLRightInline}[1]{\parbox[t]{\linewidth - \csllabelwidth}{#1}}
\newcommand{\CSLIndent}[1]{\hspace{\cslhangindent}#1}
$endif$

\begin{document}
%%%% Here you just need to include/input your actual work.
%       The above files (dedication, acknowledgement, titlepage, etc etc) will all be added for you
%       using the files you assigned above.
%       If you want to input the above files manually you can comment out the \setFILE command above
%       and use \input or \include here. Generally you want to use \include to get your pagebreak.
%       NOTE: If you input manually you will have to do some/all the formatting manually.



\chapter{INTRODUCTION and opening remarks} \label{intro}

We automatically capitalize all chapters, but if you need to suppress this you can use the class option ``overrideTitles" and/or ``overrideChapter" to allow you to use non-capitalized letters in the title and/or chapter names respectively. For more detailed information on the template's features and options, see the included file ``ufdissertation-Doc-and-Troubleshooting".

\renewcommand*{\thefootnote}{\fnsymbol{footnote}}\footnote{an un-numbered footnote - this is how you tell the readers that this chapter was previously published and then cite the Journal where it was published} We don't recommend that you change much of anything in the class file unless you're absolutely sure of what your are doing.\renewcommand*{\thefootnote}{\arabic{footnote}}\setcounter{footnote}{0}\footnote{and now we're back to normal footnote marking} 

\section{The Section Command Text Should Be in Title Case}

Title case is where all principal words are capitalized except prepositions, articles, and conjunctions.  %\cite{green2008wrinkle}

\subsection{Subsection Commands Are Also in Title Case}
The difference, of course, are the second level headings are left-aligned

\subsubsection{Subsubsections are in sentence case}
The third level subheadings are left-aligned but in sentence case. Only the first letter and any proper nouns are capitalized. %\cite{strickler1998contamination}

\subsubsection{If you divide a section, you must divide it into two, or more, parts}

{\bf Paragraph headings.} There is no official fourth level heading. Do not use the Paragraph heading feature in LaTeX, simply apply the bold characteristic to the first few words of a paragraph followed by a colon or period.

\subsection{I Need Another Second Level Heading in This Section}

Aliquam mi nisi, tristique at rhoncus quis, consectetur non mi. Phasellus blandit quam ligula, a viverra lacus commodo at. In iaculis nisl vel pretium sollicitudin. In efficitur massa vel elit sollicitudin, vel auctor sapien cursus. Proin feugiat sapien a mi tempus;

 $ X-X'=D+D'$

 in consequat augue cursus. Nulla sed sagittis purus. Nunc eu consequat orci, eu laoreet enim. Ut euismod tincidunt sem, eget lacinia dui luctus eu. Aliquam mi augue, faucibus id semper vitae, porta ac ligula. Morbi sed ultrices odio. Mauris id luctus ex. Nulla ac libero dictum, interdum turpis lacinia, scelerisque leo. Praesent varius orci ac eros varius pharetra.
% Modified from old template.

%\begin{algorithm}% Example showing the weird "algorithm" environment works...
%    \captionof{algorithm}{Test Caption}
%\end{algorithm}
%\addObject{TestStuff!}%     This is probably the command that a normal author will use to add objects.

\chapter{LITERATURE REVIEW} \label{lit}

\section{Dolor Sit Amet}

 Many of the problems in theses and dissertations involve tables. The UF Graduate Counsel is very specific in the Table Requirements.There should be no vertical lines in tables and only three horizontal lines. No bold text, etc., see the web site for the complete list of requirements. One simple improvement can be incorporated by using tabularx instead of the tabular environment. This allows a table to be stretched the full text width easily, which avoids the centered or left aligned issue. Table \ref{first} is an examble of the tabularx code. Consectetur adipiscing elit. Fusce eget tempus lectus, non porttitor tellus. Aliquam molestie sed urna quis convallis. Aenean nibh eros, aliquam non eros in, tempus lacinia justo. In magna sapien, blandit a faucibus ac, scelerisque nec purus. 
 
\begin{table}[htbp]% Fix the table captions to sit directly on the table, but figures do NOT sit directly on the figure.
    \captionof{table}{A sample Table using tabularx}\label{first}
    \begin{tabularx}{6.5in}{XXX}
      \hline
      First & Second & Third \\
      \hline
      12 & 45 & 26 \\
      17 & 32 & 93 \\
      text & 51 & can be there too. \\	
      \hline
    \end{tabularx}
\end{table}
 
 
 Praesent fermentum felis nec massa interdum, vel dapibus mi luctus. Cras id fringilla mauris. Ut molestie eros mi, ut hendrerit nulla tempor et. Pellentesque tortor quam, mattis a scelerisque nec, euismod et odio. Mauris rhoncus metus sit amet risus mattis, eu mattis sem interdum.

 \begin{table}[htbp]
    \caption{A sample Table using standard tablular}\label{first}
    \begin{tabular}{c c c}
      \hline
      First & Second & Third \\
      \hline
      12 & 45 & 26 \\
      17 & 32 & 93 \\
      text & 51 & can be there too. \\	
      \hline
    \end{tabular}
\end{table}

\subsection{Platea Dictumst}
Donec convallis scelerisque ante, in sollicitudin orci laoreet eu. Nam arcu magna, semper vel lorem eu, venenatis ultrices est. Nam aliquet ut erat ac scelerisque. Maecenas ut molestie mi. Phasellus ipsum magna, sollicitudin eu ipsum quis, imperdiet cursus turpis. Etiam pretium enim a fermentum accumsan. Morbi vel vehicula enim.

%\begin{algorithm}[!t]
%\caption{Random Forest Training}\label{alg:ranFor}
%\begin{algorithmic}[1]
%\Procedure{Train}{$X,Y$}
%\State $i \Leftarrow 0$
%\While{$i < TargetNumberOfTrees$}
%\State Sample subsets  $X_i$ and $Y_i$ from X and Y
%\State Train tree $T_i$ with $X_i$ and $Y_i$
%\State $i = i + 1$
%\EndWhile\label{ranFor}
%\State \textbf{return}  Set of Trees Trained 
%\EndProcedure
%\end{algorithmic}
%\end{algorithm}



\section{Ex id ullamcorper commodo}
Augue sapien mattis leo, nec accumsan turpis quam at neque. Ut pellentesque velit sed placerat cursus. Integer congue urna non massa dictum, a pellentesque arcu accumsan. Nulla posuere, elit accumsan eleifend elementum, ipsum massa tristique metus, in ornare neque nisl sed odio. Nullam eget elementum nisi. Duis a consectetur erat, sit amet malesuada sapien. Aliquam nec sapien et leo sagittis porttitor at ut lacus. Vivamus vulputate elit vitae libero condimentum dictum. Nulla facilisi. Quisque non nibh et massa ullamcorper iaculis.

Integer laoreet bibendum arcu non pulvinar. Curabitur ac magna nibh. Phasellus sed nisi semper, molestie neque at, tempus lacus. Aenean vitae lacinia est. Phasellus aliquam lacus sit amet placerat molestie. Sed sit amet bibendum lectus, ac ornare ligula. Curabitur porttitor interdum tortor a dignissim. Quisque a placerat nibh. Phasellus lobortis imperdiet augue, non congue est bibendum eu. Vivamus tincidunt quam eu fringilla laoreet.

Maecenas efficitur dolor et ipsum convallis, ut fringilla neque luctus. Donec ac nisl quis leo gravida accumsan sit amet sed tellus. Quisque placerat hendrerit augue sit amet aliquet. Vestibulum laoreet consequat nunc, et egestas nisl auctor et. Duis scelerisque vulputate placerat. Proin tempus ligula ac tempor eleifend. Nullam est odio, commodo quis nisl eu, feugiat efficitur purus.

Duis egestas in mauris vel efficitur. Sed a faucibus sem, non euismod enim. Maecenas nec nulla justo. Suspendisse ut orci ac mi aliquet tincidunt ac eget quam. Quisque ac mi sagittis, dapibus dui a, facilisis neque. Aenean euismod orci sem, non imperdiet ipsum pulvinar ac. Proin eu vestibulum magna, eu ullamcorper nulla. Etiam enim felis, dignissim eget commodo ac, faucibus nec justo. Nulla condimentum velit imperdiet ligula aliquam semper. Nulla facilisi. Ut in lobortis metus, at dictum ipsum. Suspendisse facilisis nec eros eget mollis. Vestibulum eget dolor ac mauris lobortis gravida. Suspendisse consectetur orci in risus pharetra, sed eleifend nisl lacinia. Mauris augue nibh, commodo sed sem at, congue molestie massa. Suspendisse sodales aliquet tellus, a tristique nunc aliquam id.

% Modified from old template.
\include{chapter3}% Modified from old template.

\chapter{EXAMPLES OF EDITOR/Author TOOLS, TABLES, AND IMAGES}% Notice that we can use chapter/section etc breaks in the master file if we want, and then use \input instead of \include to avoid unneccessary page breaks.
\input{editorAndAuthorRemarks}%     Stuff about using editorRemark and authorRemark commands
\input{includingTablesExamples}%    Stuff about using Tables.
\input{includingImagesExamples}%    Stuff about using Images.

\include{chapter5}% Modified from old template.

$body$

\end{document}

